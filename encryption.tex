\section{Encryption}

In this section, we will briefly introduce the authentication and key
generation phases, then study two important bulk data encryption
algorithms (DES and AES), followed by a brief look at other uses of
encryption techniques besides bulk data encryption.

\subsection{Authentication}

The authentication phase can be accomplished using a pre-shared secret
key and symmetric encryption of some message and
response~\cite{needham:encrypt}, or it can be done via
asymmetric, public-key cryptography~\cite{rivest78:_rsa}.

The best-known and most-used form of public key crypto is RSA, developed by Rivest, Shamir and Adelman (but also previously discovered by Cocks, of a British intelligence agency):

\begin{enumerate}
\item Pick two large primes, $p$ and $q$, let $n = pq$.
\item Calculate $\phi(p,q) = (p-1)(q-1)$.
\item Pick $e$, prime relative to $\phi(p,q)$.
\item Calculate $d$, s.t. $de = 1 \bmod \phi(p,q)$,
  which we can call the inverse of $e$ modulo $\phi(p,q)$.
\end{enumerate}

The tuple $\langle n,e\rangle$ is now the public key, and $d$ is the
corresponding private key.  The creator can publish $\langle n,e\rangle$ any
way they like, including in the New York Times. To send the creator a message
$m$, the sender calculates the ciphertext $c$,
\begin{align}
c = m^e \bmod n
\end{align}
and send $c$ to you.  You can recover the message $m$ by
\begin{align}
m = c^d \bmod n.
\end{align}

\comment{informal}
The drawback to public key cryptography is that it is expensive, so it
is not used for every message.  If I send you "Hi, Bob, it's Tuesday,
let's use 42 as our key," using your public key and you reply,
"Tuesday is beautiful, Alice, and 42 is fine," using my public key, we
confirm that we are definitely Bob and Alice, known as authentication
-- assuming, of course, that you trust that the key $\langle
n,e\rangle$ that you are holding really and truly belongs to me, and I
haven't leaked it out somehow!

As the publication of $n$ makes clear, RSA is vulnerable to advances
in factoring larger integers, hence interest from intelligence
agencies and security researchers was immediate when Shor's algorithm
was announced.

Current EU recommendations are for 3072-bit RSA keys, recently (2018)
upgraded from 2048, for "near-term protection" (up to ten years).
They recommend an extraordinary 15360 bits for "long-term protection"
(thirty to fifty years). [Cloudflare, Keylength.com, Ecrypt, Lenstra]

Finding prime numbers large enough to be used in RSA also requires a
substantial amount of computation~\footnote{See
  \url{https://en.wikipedia.org/wiki/Primality_test}.}.

[called ECRYPT-CSA 2018, via Shigeya]

\subsection{Key generation}

The other algorithm often mentioned as being both important and
vulnerable to Shor's factoring algorithm is Diffie-Hellman key
exchange, which is used for creating the session key we will use for
bulk data encryption~\cite{diffie1976new}.  It is important that every
communication session have its own separate encryption key;
Diffie-Hellman provides a fairly efficient means of creating these
keys in a distributed fashion.

D-H works as follows:

\begin{enumerate}
  \item Alice and Bob publicly agree on a modulus $p$ and a base $g$ (in cleartext is okay)
  \item Alice and Bob each pick a secret random number $a$ and $b$
  \item Alice calculates $A = g^a \bmod p$, \\
    Bob calculates $B = g^b \bmod p$
  \item Alice sends Bob $A$, \\
Bob sends Alice $B$ (in cleartext is okay)
  \item Alice calculates $s = B^a \bmod p$, \\
	   Bob calculates $s = A^b \bmod p$
\end{enumerate}

Both Alice and Bob now have the same secret $s$, which they can use as
an encryption key.

Note that, as-is, D-H is vulnerable to a man-in-the-middle attack, and
so must be coupled with some form of authentication so that Alice and
Bob each know that the other is who they say they are.

\subsection{Bulk data encryption}

D-H and RSA are pretty easy to understand, and known to be vulnerable
to quantum computers, hence attract a lot of attention. The workhorse
symmetric block ciphers for bulk encryption are actually much more
complex mathematically, and hence harder to understand, but ultimately
can be executed efficiently on modern microprocessors and dedicated
chips.

A block cipher takes the data to be encrypted, and breaks it into
fixed-size chunks called blocks.  If the last block isn't full, it is
filled out with meaningless data.  We also take a key, and using the
key perform mathematical operations on the data block.  In a symmetric
system, by definition, the decryption key is the same as the
encryption key.  Generally, the output block is the same size as the
input block.  There is no requirement that the block and key are the
same size.

Ideally, the output block (the ciphertext) will look completely
random: about 50\% zeroes and 50\% ones, regardless of input, and with
no detectable pattern.  That is, its entropy will be very high.  Of
course, it cannot be completely random, as it must be possible for the
data to be decrypted by someone holding the appropriate key.  A good
cipher, however, will provide few clues to an attacker.  A single
bit's difference in either the key or the original plaintext should
result in about half of the ciphertext bits being flipped, so that
being "close" offers no guidance on a next step.

Thus, a symmetric key system's security is often linked to its key
size; with $k$ key bits, it should require, on average, $2^{k-1}$
trial decryptions to find the key and to be able to decrypt the
message.  We will discuss this further when we get to cryptanalysis.

Many encryption algorithms have been designed over the years, but two
in particular are too important to ignore, so we will examine them:
DES, the Data Encryption Standard, which is still in use in modified
form but is primarily of historical interest now, and AES, the
Advanced Encryption Standard, which is used for most communications
today.

\subsubsection{DES (the Data Encryption Standard)}

\begin{figure}
  {\color{Magenta} Add a figure with the standard DES picture.}
  \caption{The DES encryption process.}
  \label{fig:des}
\end{figure}

DES, the Data Encryption Standard, was designed by IBM in the 1970s,
consulting with the NSA~\cite{standard199946}.  DES is a type of cipher known as an
iterative block cipher, which breaks data into to blocks and repeats a
set of operations on them.  The operations in DES are called a Feistel
network, after the inventor.

DES uses a 56-bit key, though products exported from the U.S. were
limited to using 40 meaningful key bits [wikipedia/40-bit-encryption].
It was later upgraded to triple-DES, using three 56-bit key pieces and
repeating DES three times, giving up to 168 bits of protection.
However, in a block cipher, not only does the key size matter, the
block size also matters, as we'll see below.  For DES, that block size
is only 64 bits.

DES operates in sixteen rounds, each of which uses a 48-bit subkey
generated from the original 56-bit key using a key scheduling
algorithm.  In each round, half of the block is tweaked and half
initially left alone, then XORed with the tweaked half.  The two
halves are swapped before the next round.

The "tweaking" of the right half of the block is done by first
expanding the 32 bits into 48 by replicating half of the bits, XORing
with the 48-bit subkey, then dividing it into 6-bit chunks and pushing
each chunk through one of eight substitution boxes, or S boxes.  Each
S box turns 6 bits into 4, using a lookup table defined as part of the
algorithm (that is, this operation is not key-dependent).  \aono{} The S boxes
are nonlinear (but not affine), which is the source of the true
security of DES; if the S boxes were linear, breaking DES would be
easy (or so I am told).

Decrypting DES is exactly the same operation as encrypting, except
that the subkeys are used in reverse order.

More formally, the sequence of operations in a DES encryption
is:

\begin{enumerate}
\item Apply initial permutation (IP) (a fixed operation)
\item For $i = 1$ to 16 do
  \begin{enumerate}
  \item divide the block into two 32-bit halves
  \item expand the left half to 48 bits (a fixed operation)
  \item calculate subkey $i$:
    \begin{enumerate}
    \item split key into two 28-bit halves
    \item rotate each half 1 or 2 bits (a fixed operation according to
      the key schedule)
    \item select a subset of 48 bits (a fixed operation according to
      the schedule)
    \end{enumerate}
  \item XOR subkey $i$ with the left half of the block
  \item split the left half into eight 6-bit pieces
  \item push each 6-bit piece through a $6\rightarrow 4$ S-box
  \item permute and recombine the eight 4-bit pieces (a fixed operation)
  \item XOR the left half with the right half of the block
  \item swap halves of the block
  \end{enumerate}
\item Apply the final permutation (FP) (a fixed operation)
\end{enumerate}

The $6 \rightarrow 4$ S boxes are obviously inherently non-reversible,
but the earlier expansion guarantees that ultimately no information is
lost as the block passes through the entire network.

\comment{Should this be in the cryptanalysis section?}  The success of
a cryptanalytic technique is often measured in terms of the number of
rounds it can break in a multi-round cipher like DES.  For example, if
DES were six rounds instead of sixteen, the technique can find a
subkey or undo the encryption.  Various attacks have been able to
penetrate DES to different depths.  Of course, the more
straightforward approach is sheer brute force; as early as 1977,
Diffie and Hellman published a critique in which they argued that
brute force stood at the edge of feasibility~\cite{diffie1977special},
and indeed in 1999, a special-purpose machine named Deep Crack, built
by the Electronic Freedom Foundation, cracked a DES key.  Further
advances have made it possible even for dedicated hobbyists to crack.\comment{cite}

Triple-DES (also called 3DES) has several modes of operation, but is
usually used with three independent 56-bit keys, $K1$, $K2$, and $K3$,
with encryption performed as $C = E_{K3}(D_{K2}(E_{K1}(P)))$ where $P$
is the plaintext, $E$ and $D$ are the encryption and decryption
operations, and $C$ is the ciphertext.

DES was withdrawn as a standard in 2005, after having been replaced by
AES in 2001, although the U.S. government still allows 3DES until 2030
for sensitive information. \comment{cite}

\subsubsection{AES (the Advanced Encryption Standard)}

\begin{figure}
  {\color{Magenta} Add a figure with the standard AES picture.}
  \caption{The AES encryption process.}
  \label{fig:aes}
\end{figure}

AES, the Advanced Encryption Standard, has replaced DES.  It is a type of block cipher known as a \emph{substitution-permutation} cipher.  AES uses a block size of 128 bits, and a key size of 128, 192 or 256 bits. The key size affects the number of rounds of substitution-permutation, requiring 10, 12 or 14, respectively.

AES began life in 1997 when NIST announced a competition for a successor to DES, due to the problems with DES described earlier and later. Two Belgian cryptographers, Vincent Rijmen and Joan Daemen, submitted a proposal they dubbed Rijndael, which ultimately beat out fourteen other serious competitors to become, with some modifications, the Advanced Encryption Standard, published as the standard in 2001.

A substitution-permutation network alternates substituting new bits
for old ones with permuting (rearranging) the bits, conducted over a
series of rounds. DES has some characteristics of a
substitution-permutation network, but is not purely so. A
well-designed s-p net will maximize Shannon's confusion (or the
avalanche effect) and diffusion, which we will see shortly
(Sec.~\ref{sec:diffusion}).

AES begins by laying out the sixteen bytes in a $4\times 4$ array in
column major form: the first byte in the upper left, then the next
byte below it, with the fifth byte returning to the top of the second
column. Most operations work on rows, however, increasing the mixing
of the bytes.

First, in the KeyExpansion phase, the round keys are created from the original key.

Next, the appropriate number of rounds is performed. Each round consists of the following steps:

\begin{enumerate}
\item SubBytes
\item ShiftRows
\item MixColumns
\item AddRoundKey
\end{enumerate}

In the first round, AddRoundKey is performed before the other operations as well as after. In the last round, the MixColumns step is omitted.

The S box in SubBytes is a single, byte-level, fixed operation, $1:1$,
unlike the DES S boxes, which are more complex and vary depending on
the position within the block. The AES S box is nonlinear ($f(x+y) \ne
f(x)+f(y)$) and was designed specifically to defeat both linear and
differential cryptanalysis (which we will see in
Sec.~\ref{sec:cryptan}), but should the implementer or user not trust the specific values they can be modified. The S box can be implemented as a straightforward 256-byte lookup table, with the input byte as the index, or as a simple bitwise matrix multiplication.

The ShiftRows operation rotates all rows but the first. The MixColumns operation is more complex, executed as a matrix multiplication using modulo-two integer arithmetic. These two operations maximize the diffusion of the original data. The AddRoundKey is a simple XOR of the 128-bit round key into the block.

\comment{This was objected to.}
AES is easily implemented in either software or hardware. Some modern
CPUs include special instructions to accelerate AES
encryption~\footnote{On recent Intel processors the instructions
  \textsc{aesdec}, \textsc{aesdeclast}, \textsc{aesenc},
  \textsc{aesenclast}, \textsc{aesimc}, and \textsc{aeskeygenassist}
  are custom-created for AES, and \textsc{pclmulqdq} is a more generic
  instruction for block-oriented encryption.} , and even for those that
don't, heavily optimized assembly language is achievable.  In the
absence of special hardware, encrypting each 128-bit block of data
requires several hundred CPU instructions. Example encryption rates
for a modern laptop are shown in Tab.~\ref{tab:mac-aes}

% \begin{widetext}
% \begin{verbatim}
% 128-bit key: 1.25 Gbps for 16-byte chunks, 1.44 Gbps for 8192-byte chunks
% 192-bit key: 1.06 Gbps for 16-byte chunks, 1.18 Gbps for 8192-byte chunks
% 256-bit key: 912 Mbps for 16-byte chunks, 1.01 Gbps for 8192-byte chunks
% \end{verbatim}
% \end{widetext}

\begin{table}
\begin{tabular}{l | r | r}
            & 16-byte chunks & 8,192-byte chunks \\\hline
128-bit key & 1.25 Gbps & 1.44 Gbps \\
192-bit key & 1.06 Gbps & 1.18 Gbps \\
256-bit key & 912 Mbps & 1.01 Gbps
\end{tabular}
\caption{AES-CBC encryption rates on a Macintosh laptop with a 2.2GHz
  Intel Core i7 processor, taken using the {\tt openssl speed} test
  program (see App.~\ref{sec:speed-test}).}
\label{tab:mac-aes}
\end{table}

Despite two decades of effort, there are no known attacks on AES that
allow recovery of the key in substantially less time than brute force;
the best attacks roughly halve that cost, at the cost of requiring
tremendous amounts of storage. \comment{cite}

\subsubsection{Limitations of Block Ciphers}

One problem with the most straightforward application of a block
cipher is that it is deterministic.  The transform can be applied to
each block independently; this is known as \emph{electronic code book}
(EBC) mode.  However, if the same ciphertext $c$ is observed in two
different places in the message stream, the attacker knows that the
two input blocks were the same.  This fact leaks information to a
sophisticated attacker, and is considered unacceptable.

One answer to this is to make the encryption slightly less
deterministic by XORing in the ciphertext of the \emph{previous} block into
the current one before performing the encryption.  This is cipher
block chaining (CBC) mode, the standard mode of operation.  (There are
at least two more modes that I know nothing about.)  CBC has the
undesirable side effect of requiring encryption to be done serially,
but attacks can be parallelized.  Nevertheless, it's the most commonly
used mode.

\subsection{Other Tasks}

\comment{Crypto also used for passwords; signatures; integrity hashes (both
  signed ans unsigned); spread spectrum for secrecy, difficulty of
  triangulation and being hard to jam; challenge response rfid and
  smart credit cards.}

\comment{bitcoin here or elsewhere?  Blockchain is an \emph{auditable
    communication channel}~\cite{suzuki2017blockchain}.}

\comment{Internet security-oriented encryption not covered yet: ssh/scp, DNSSEC, BGPsec}

\subsection{Notes \& References}

To be added.

