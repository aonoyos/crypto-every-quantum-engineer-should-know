\section{Attacking classical cryptography using quantum computers}

Naturally, my own interest in writing these notes stems from my
experience in quantum computing and quantum networking.  The previous
sections dealt primarily with classical attacks, with a little bit of
quantum networking can be integrated with classical thrown into each
section.  Here, let's look at the attacks on classical cryptography
using quantum computers.

\subsection{Shor 'Nuff}

Obviously, the main thing we're talking about here is Shor's
algorithm.  I have not been privy to the conversations of
cryptographers in creating post-quantum crypto, though we'll take a
short look at that below.  But there are very few people in the world
who understand better than I do what it takes to actually run the full
version of Shor's algorithm on a potentially real machine.

A full implementation of Shor's algorithm consists of two quantum
phases: modular exponentiation, followed by the quantum Fourier
transform.  (I'm assuming you're familiar with the main ideas behind
Shor, how it uses interference to build patterns in the quantum
register that can reveal the period of a function that allows us to
find the factors of a large semi-prime number.)

The key insight may be the behavior of the QFT, but the bulk of the
execution time will be in the modular exponentiation phase.  This is
actually one of the areas that I worked on in my
Ph.D. thesis~\cite{van-meter06:thesis}.  Some of the important parts
of this were published in Physical Review
A~\cite{van-meter04:fast-modexp}.

In my thesis there is a plot that I think is useful, which we updated
and published in our Communications of the ACM article~\cite{van-meter13:_blueprint}.

We worked out detailed performance estimates, including hardware
requirements and quantum error correction, in some of our papers~\cite{van-meter10:dist_arch_ijqi,PhysRevX.2.031007}.

There were other, contemporary important papers on how to implement
Shor, including Fowler et al.~\cite{fowler04:_shor_implem},
who discussed Shor on a linear array, a few months ahead of my own
Phys. Rev. A paper on the same topic.

We both built on important, early work by Vedral, Barenco and Ekert
(VBE)~\cite{vedral:quant-arith} and by Beckman, Chari, Devabhaktuni and Preskill (BCDP)~\cite{beckman96:eff-net-quant-fact}.

If you are looking to broaden your reading on implementations of
Shor’s algorithm, the following are useful:

Pavlidis and Gizopoulous found an efficient division algorithm,
accelerating the math~\cite{Pavlidis:2014:FQM:2638682.2638690}.

Roetteler, Steinwandt focused on the applicability of Shor's algorithm
to elliptic curve. I like the paper, except I think their survey of
related research could have been better.
\url{https://arxiv.org/abs/1306.1161}
and Roetteler, Naehrig, Svore, Lauter:
\url{https://arxiv.org/abs/1706.06752}

Gidney and Ekera submitted to the arXiv a paper on factoring a
2048-bit number using 20 million noisy quibits.  In this paper, one of
their techniques is arithmetic "windowing" which is essentially
identical to one of the techniques I proposed in my 2005 Phys. Rev. A
paper.  \url{https://arxiv.org/abs/1905.09749}

May and Schliper also recently uploaded a paper on period finding with
a single qubit.
\url{https://arxiv.org/abs/1905.10074}

Ekeraa and Hastad also proposed a new period-finding algorithm
variant, in 2017.
\url{https://link.springer.com/chapter/10.1007/978-3-319-59879-6_20}
\url{https://arxiv.org/abs/1702.00249}

Smolin, Smith, Vargo wrote something of a warning about inferring too
much from very simple demonstrations on a few qubits.
\url{https://www.nature.com/articles/nature12290}

Here is one early paper on how to assess errors in Shor’s algorithm,
by Miquel, Paz and Perazzo~\cite{miquel1996fdq}.

Chuang et al., also published an early examination of decoherence in
factoring~\cite{chuang95:_factoring-decoherence}.

Among random things, there is Dridi and Alghassi,
factoring on D-Wave; I don’t think this paper tells us very much about factoring at scale~\cite{dridi2017prime}.

That's more than enough for now, since it isn't really the focus of
what we're after here, anyway.

\subsection{Grover's amplifier}

When learning about linear cryptanalysis (above), I naturally wondered
I wonder how hard it would be to set up amplitude amplification for
the bias in LC.  The bias we are looking for is certainly small.

It turns out people have addressed that question:
\url{https://arxiv.org/abs/1510.05836}
\url{https://link.springer.com/article/10.1007/s11128-015-0983-3}
\url{https://link.springer.com/chapter/10.1007/978-3-662-48683-2_5}

One new one I hadn't seen before:
Bernstein, Biasse, Mosca on low-resource quantum factoring, using
Grover's algorithm and NFS:
\url{https://research.tue.nl/en/publications/a-low-resource-quantum-factoring-algorithm}
\url{https://cr.yp.to/papers/grovernfs-20170419.pdf}

\subsection{Notes \& References}

To be filled in eventually, mostly by moving the existing parts of
this section into this subsection!

