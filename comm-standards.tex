\section{Standards for Encrypted Communication}

\subsection{IPsec and the IETF}
\label{sec:ipsec}
\begin{figure}
  {\color{Magenta} IPsec picture; network-to-network, as well as hub-and-spoke for ``road warrior'' clients.}
  \caption{The DES encryption process.}
  \label{fig:ipsec}
\end{figure}

This paper was inspired in part by a discussion of the performance
demands made of key generation and rekeying procedures, in the context
of incorporating QKD as well as taking into consideration Shor's
algorithm.  In this section, we return to that original topic and
answer several questions:

\begin{enumerate}
\item What are the technical mechanisms in place for rekeying and
   effective use of encryption in the Internet standard security
   protocol IPsec?
\item What was known, at the time these protocols were developed, about
   the best practices for rekeying?
 \item What are best practices today?
   \item In light of these topics, what are performance demands for
     key generation or rekeying?
\end{enumerate}

To answer these, we begin with a discussion of IPsec itself, and
introduce the organization through which the protocols are created.

\subsubsection{Background on IPsec, IETF and RFCs}

The Internet Engineering Task Force (IETF) is where protocol
specifications for the Internet come from.  There is an entire Area
within IETF (an "area" is the largest size organizational group in
IETF) dedicated to security, which charters many different working
groups~\footnote{\url{https://trac.ietf.org/trac/sec/wiki}.}.
Security is much more than cryptography alone, but an important area
of work is developing the network protocols that allow real systems to
use the cryptographic techniques discovered by the mathematicians.
Moreover, theorists are inevitably naive about how much work it is to
actually use their ideas.

One of the most important means of securing your communications is
IPsec, which builds a "tunnel" inside of which ordinary IP packets can
be carried transparent to their origin and destination (meaning your
laptop and the server don't have to be be configured to handle the
encryption; they deal in unmodified, unencrypted IP packets) but
protected as they transit public networks.

IPsec is complex and has been updated many times.  The Wikipedia page
on it (which might be an easier entry point than the IETF indices,
which are organized chronologically) lists over 40 standards-track
documents, probably totaling over a thousand pages, some of which are
outdated and some of which are still
current~\footnote{\url{https://en.wikipedia.org/wiki/IPsec}.}.

Those documents are what are known as RFCs, or Request for Comments
documents.  They have different levels of authority, ranging from
Experimental and Informational to Standard.  Reaching Standard can
take decades and numerous iterations as the working groups gradually
converge on what works in the real world, intersecting with what
people will actually implement and use, but protocols are often de
facto standards long before reaching that platinum frequent flyer
status.

\subsubsection{IKE and IPsec}

If you really want to dig into the key exchange protocol, RFC 7296
(Oct. 2014) is the most modern reference~\cite{RFC7296}.  Unless
you're actually manually configuring or implementing the stuff, you
probably won't care about the differences between it and the older
versions.

But you might want to start with RFC 4301 (Dec. 2005) (also a proposed
standard), which is titled, "Security Architecture for the Internet
Protocol"~\cite{RFC4301}.

IPsec has a couple of modes, but let's stick to what's called tunnel
mode.  Two boxes, known as gateways, build one or more Security
Associations (SAs). An SA describes which packets passing between the
gateways are to be encrypted and how.  Those that are encrypted are
encrypted in their entirety (packet headers and all), and sent as the
payload of another IP packet, to be decrypted at the far end.  Tunnel
mode is most often used to connect together via the Internet two
networks (e.g., two offices of the same company) that are each
considered to be relatively secure networks.  The packets between
computers in one network and those in the other network are encrypted
only during their transit from gateway to gateway.  Of course, these
days, much (most?) data is also encrypted by the end hosts, especially
for the two major applications of web and email, so much of the
traffic in the tunnel will be double-encrypted.

The first SA created is the IKE SA itself, used only to carry the
messages that govern the tunnel.  The first exchange of messages
negotiates some security parameters, and carries random nonces used to
"add freshness" to the cryptographic exchange and the parameters to be
used for the Diffie-Hellman key exchange.  \ddp{} I believe this is where the
preferred choice for the bulk encryption (3-DES v. AES v. whatever) is
also negotiated.  Since we have not yet established the tunnel, these
messages are necessarily sent as plaintext.

A block of material called SKEYSEED is calculated independently by
both ends using the nonces generated by both ends and the
shared secret generated by the Diffie-Hellman exchange in the \ddp{} INIT.
Building SKEYSEED involves the use of a pseudorandom function (PRF)
also agreed upon...in the first exchange?  I'm having trouble tracking
where that's chosen.

SKEYSEED is used first to generate a key for the next message
exchange, and then later to make keys for the Child SAs (below).

Next, there is an encrypted exchange that is used to authenticate the
parties.  The authentication may be via an RSA digital signature, a
shared (symmetric) key message integrity code, or a DSS digital
signature~\cite{dss}.  In all three methods, each party signs a block of data
using the secret, in a fashion that can be verified by the partner.
(This could again be vulnerable to Shor's algorithm if it uses one of
the public key methods, but keep in mind the messages containing this
information are also encrypted; however, as we are just now
authenticating, it's \emph{possible} that, up to this point, the partner at
the other end of this connection is not who they claim to be!)

The IKE SA is used to create Child SAs, which carry the actual
traffic.  The keys used for the Child SAs, therefore, are the obvious
target for traffic-based attacks, though the real prize is the keys
for the IKE SA.  I'm having a hard time imagining how to mount an
effective attack against the IKE SA.

The key material for the Child SA is generated via a complex mechanism
involving a new nonce and the PRF previously specified.  The initiator
of the creation may also, optionally, specify that an entirely new
Diffie-Hellman exchange be performed.  I'm very unclear on how often
that option is used in practice.

Each SA, whether IKE or Child, can (and should) have a lifetime.  That
lifetime can be specified in either seconds or in bytes that have been
encrypted as they pass through the tunnel.  Once the lifetime has
expired, the two gateways must create a new Child SA with new keys.
This ultimately is the heart of what we're looking for here: what is
that recommended lifetime today, and what should it be in the light of
quantum computing?

\subsubsection{Digging into the Cryptanalysis of IPsec}
\label{sec:ipsec-cryptan}

\outlinecomment{Should part of the email search be relegated to an appendix?}

\comment{informal}
What is the recommended lifetime of an IPsec Security Association
today?  This is the question that has proven to be so hard to answer,
and that has led me wandering all across the web.  Probably the most
relevant source is, naturally, the mailing list where most of the
design work is documented.

One early, relevant message
(\url{https://mailarchive.ietf.org/arch/msg/ipsec/_dSgUvW6WiUFvw5aoyu7rJo5Kgc})
from 08 March 1995 carries the quote:

\begin{widetext}
\begin{verbatim}
	"I think 2^32 is a better bound than 2^43, at least for certain modes
	of DES. For instance, after 2^32 blocks in CBC mode, you expect to see
	two identical ciphertext blocks, say c[i] and c[j]; the difference
	between their predecessors will match the difference between the
	corresponding plaintext blocks, i.e.,

	p[i] xor p[j] = c[i-1] xor c[j-1]

	Information thus starts to leak after 2^32 blocks (square root of the
	message space). I would recommend 2^32 blocks as the limit for the
	lifetime of a key, and that takes care of the 2^43/2^47 attacks as
	well."
\end{verbatim}
\end{widetext}
referring, although not by name, to both the birthday paradox and the
differential cryptanalysis limits discussed above.  Keep in mind that
at $2^{32}$ blocks, we are at a 39\% probability of there being at least
one ciphertext collision revealing some information.

Searching the archives for "birthday" also turned up some relevant
messages, e.g. the relatively recent (21 April 2015) message
\url{https://mailarchive.ietf.org/arch/msg/ipsec/T1woQuwh1Ccoz6fWWFDBETBllaY}
quoting the earlier message
\url{https://mailarchive.ietf.org/arch/msg/ipsec/De46HNR1h4lGXZnxlIVpNSfs5p0}

\begin{widetext}
\begin{verbatim}
	"> I think the main problem with 3DES is not that it is significantly slower
	> than AES, but that it has blocksize of 64 bits, that is considered
	> loo small for high-speed networks, when the possibility of birthday attack
	> leads to necessity to frequently rekey.

	It’s hard to make that case. The blocksize is 64 bits. So it’s prudent
	to not use more than, say, a billion blocks. A billion blocks is 64
	Gb. There are very few real tunnels that run that kind of throughput
	in under a minute. OTOH it’s no problem at all to run a CreateChildSA
	every minute, or even every five seconds. So I think there are very
	few cases that *can’t* use 3DES."
\end{verbatim}
\end{widetext}

This is interesting, particularly given its newness.  The author (Yoav
Nir, one of the long-time leaders of the IPsec community) considers
3DES plus very frequent rekeying to be sufficient, at least for some use
cases, and it's important for backwards compatibility.  However, in a
slightly earlier (2012) exchange on the mailing list, David McGrew
(another key IPsec person) and Nir covered the same issue, with McGrew
arguing that no more than 50 megabytes should be encrypted with the
same key even using 3DES, due to the birthday paradox.  McGrew went so
far as to write up a 16-page analysis posted on the Cryptology
preprint server~\cite{cryptoeprint:2012:623}.

\comment{Need a summary here, and tie it back to the intro, where I said ``half an hour''.}

\subsubsection{IPsec, Quantum Computing and QKD}

\comment{Papers on actually integrating QKD into classical crypto~\cite{Alleaume201462,mink09:_qkd_and_ipsec}.}

\comment{I have two major QKD reviews in my stacks somewhere, haven't
  found either yet...~\cite{Alleaume201462} isn't one, but is
  related...}

\comment{Xu-Ma-Zhang-Lo-Pan:
https://arxiv.org/abs/1903.09051
Pirandola-et-al:
[1906.01645] Advances in Quantum Cryptography
}

\comment{informal}
(This one is just a placeholder, but there is surprisingly little
\emph{documented} about this kind of work.  Even e.g. the publications
describing the recent Chinese QKD satellite
experiment~\cite{PhysRevLett.120.030501} say, "We used AES plus QKD,"
or, "We used QKD-generated keys as a one-time pad," but they tell you
nothing about the \emph{network} or \emph{transport} layer protocols
used, so who knows?  Might be documented elsewhere on the web, or
described in talks, but so far I've failed to track down anything
authoritative.)

Chip Elliott was the first to modify IPsec to work with QKD, as far as
I'm aware, described in a 2003 SIGCOMM paper~\cite{elliott:qkd-net}.
Shota Nagayama drafted and implemented a more formal protocol
modification for his bachelor's thesis working with me, back in 2010,
but we failed to push that from Internet Draft to
RFC~\cite{nagayama12:_ike_for_ipsec_with_qkd}.

\comment{Also~\cite{tanizawa2016secure}.}

(More to come here, eventually, on both the use of QKD with protocols
such as IPsec and on how effective quantum computers will or won't be
at breaking communication security.  Some of this will also be covered
in Section 5.)

\subsection{TLS/SSL and cryptography}
\label{sec:tls}

Here we want to answer the same three questions we had above:

\begin{enumerate}
\item What are the technical mechanisms in place for rekeying and
   effective use of encryption in TLS?
\item What was known, at the time this protocol was developed, about
   the best practices for rekeying?
\item What are best practices today?
\end{enumerate}

but this section will be much shorter, as I know so much less about
TLS (despite the fact that it is the most important use of encryption
on the Internet today).

The current standard for Transport Layer Security, or TLS, is
RFC8446~\cite{RFC8446}, published in 2018.  It specifies version 1.3
of the protocol.  The main document itself is only 160 pages, not bad
for something so complex and important.  (The 1.2 spec was only about
100 pages, so it has grown substantially.)

...oh, what is TLS?  It's the protocol that HTTPS runs over.  It's the
successor to SSL, the Secure Sockets Layer, which was the first way to
encrypt web browsing.  TLS originally built on top of a reliable
transport protocol, such as TCP, though a later adaptation~\cite{RFC6347}
lets it run over a datagram protocol.  We'll only worry about running
it over TCP here.  TLS provides privacy, by encrypting all of your web
browsing traffic for a particular connection.  It also provides data
integrity, using a MAC (message authentication code) when necessary.
(IPsec also does both, but we ignored that above.)

\subsubsection{TLS Records and Basic Limits}

TLS consists of one primary protocol: the TLS Record Protocol.  On top
of the TLS Record Protocol sit four protocols, one of which (the
application data protocol) is used for the bulk data encryption, and
the TLS Handshake Protocol, which utilizes public key cryptography to
establish identity, securely creates a shared secret to be used for
the bulk data encryption, and makes sure that this part of the process
can't be modified by an attacker.  A third one of interest to us is
the change cipher spec protocol.

A record in the record layer has a maximum size of 16KB ($2^{14}+1$, actually).

There is, as best I can tell, no official limit on the key lifetime or
on the number of bytes that can be pushed through a single TLS except
the limit on record sequence numbers of $2^{64}-1$.  Combined with the
max record size, that's $2^{78}$ bytes, or 256 exabytes, which is a bloody
lot of data.  So, if the lifetime of a session needs to be limited, it
has to be done by breaking the connection and renegotiating.
Apparently, adding a key renegotiate feature was considered for
TLS 1.3, but doesn't seem to have been included.

Luykx and Paterson wrote up a short (8 page) summary recommending
limits for TLS with different cryptosystems. (My copy is dated
Aug. 2017, but the paper was referred to in Apr. 2016.)
\url{http://www.isg.rhul.ac.uk/~kp/TLS-AEbounds.pdf}
Unfortunately, that paper has good references but doesn't go through
the complete reasoning, so going one level deeper in the reading is
really necessary.

In April 2016, Mozilla moved to limit the length of a connection,
based on those results:
\url{https://bugzilla.mozilla.org/show_bug.cgi?id=1268745}
They set AES-CBC to about $2^{34.5}$, or about 24 billion records, about
400 terabytes max.  That should give a probability of $2^{-57}$ of
"ciphertext integrity [being] breached", though I'm a little unclear
on exactly what that means here -- is this just birthday bound leaking
some plaintext, or is this compromise of the entire session key?
Figuring that out will take more digging, since they express things
differently than most of the resources I used above.

\subsubsection{Keying and Rekeying}
\label{sec:tls-rekeying}

\comment{how the initial key is generated}

\comment{add more here on chains of trust for certificates, and how we make
sure that an attacker can't cause a downgrade of security relative to
what both ends really want.}

\comment{how rekeying is handled}

\comment{Need a summary here, and tie it back to the intro where I said ``half an hour''.}

\subsubsection{Other Attacks on TLS}

One startling and potentially useful document is RFC7457, "Summarizing
Known Attacks on Transport Layer Security (TLS) and Datagram TLS
(DTLS)"~\cite{RFC7457}.  It details a couple of dozen attacks on TLS, most having to do
with protocol implementation issues such as failing to correctly
verify all of the information you are given.  Some of them leak
information or allow a connection to be take over, others allow an
attacker to downgrade the security negotiation so that a later attack
on the cipher has a higher chance of success.

One attack this RFC points out is
\url{http://www.openssl.org/~bodo/tls-cbc.txt}, which (among other things)
describes a flaw in how TLS Record Protocol records are concatenated.
Originally, TLS kept the ciphertext of the last block of a record to
use as the initialization vector (IV) of the first block of a new
record, but this allows an attacker who can adaptively choose
plaintexts to more completely choose the text being encrypted.  The
fix is straightforward (toss in an extra block of nonsense data before
starting the new record), and current versions of TLS aren't
vulnerable.

\subsubsection{TLS and QKD}

Alan Mink, Sheila Frankel and Ray Perlner worked on integrating TLS
with QKD, a full decade ago~\cite{mink09:_qkd_and_ipsec}.

\comment{Performance requirement: interactivity means a few seconds or less; 300ms per click, is it?}

\subsection{Other Internet Security Protocols}

\outlinecomment{Get Woolf to contribute here.}

\comment{Internet security-oriented encryption not covered yet: ssh, DNSSEC, BGPsec}

\subsection{WiFi}

\outlinecomment{Get yume or maybe koluke to contribute here?}

\subsection{Cellular Phones}

\outlinecomment{Who can contribute here?}

\subsection{Notes \& References}

To be filled in eventually.

