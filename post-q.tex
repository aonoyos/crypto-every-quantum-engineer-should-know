
\clearpage 

\section{Post-quantum cryptography}

\subsection{Motivation of post-quantum cryptography}

Since Shor's groundbreaking work, 
it has been known that a reasonable size quantum circuit can be used as 
an efficient oracle to solve a wide class of hidden subgroup problem,
which is an important generic problem that captures 
the integer factoring problem (ILP) and the discrete logarithm problem (DLP). 
As a direct consequence, developed quantum computers will make 
the most famous public-key cryptography such as RSA and DSA worthless.

As the countermeasure, the cryptographers have been developing 
new cryptographic schemes which they call ``post-quantum cryptography''
or ``quantum-resilient cryptography.'' 
It is widely believed that revealing secret information encrypted by a scheme is an intractable task even we have a large size of classical and quantum computers.

A resilience of cryptography to the computers can be explained in the following way.
The mathematical notion of security has been formulated by Shannon's discussion on the symmetric cryptography in \cite{Sha49}.
He defined the symmetric cryptography has the perfect security (information-theoretic security) if the eavesdropper that gets ciphertext will gain no information about the plaintext.
This security notion can be described as the equivalent condition that 
uses Turing machines (TM):
for any (unlimited time and space) TM $\mathcal{G}$ that outputs a candidate of plaintext from a given ciphertext, there exists a TM $\mathcal{G}'$ with no input that satisfies
%
\begin{equation}
	\label{eqn:ITalg}
	\Pr[ \mathcal{G}(ct) = m ] = \Pr [\mathcal{G}'() = m ].
\end{equation}
%
Here, the probability is over a fixed distribution of plaintext $m$ and key $k$, and $ct = Enc(k,ct)$.
The equivalence of probabilities means that $\mathcal{G}$ needs not to use $ct$ to recover $m$, thus, $ct$ does not contain any information about $m$ intuitively.

Shannon proved that the information-theoretic security 
requires that the sender and receiver have to share 
the secret key longer than the message itself.
The source of such unwieldiness is from the tightness of perfect security that requires the eavesdropper can get no information, or it allows unlimited power to the Turing machines.
It is easy to see any public-key cryptography does not satisfy
the perfect security \cite[Sect. 5.5.2]{GolEnc2},
and the followers tried to relax the notion.

One of the relaxed notion, which is commonly referred as the semantic security against classical computers is obtained by relaxing TM in (\ref{eqn:ITalg}) to a probabilistic polynomial TM (PPTM) and by some modifications \cite{...}, and security against quantum computes is obtained via a similar way.

\subsection{Classification of Post-quantum cryptographes by the base problems}

As we explained above, each security notions are defined over each cryptographic system, i.e., developers have to prove their cryptographic scheme has some property of security after the scheme proposing.
However, the rigid proof is difficult since we have to consider all types of algorithms (albeit they are limited with a polynomial) to show to have a  security notion of the scheme.

To solve the difficulty, a reduction technique has been used.

 - TBW
 
 - Classification of schemes by the base problems (lattice, code, multivariate, isogeny, etc.)

\subsection{Competition in NIST/CACR}


\subsection{Applications?}

 - Homomorphic encryption?

%�E�Í����͗ʎq�v�Z�@�������ɂł��ł���Ɖ��肵�Ă���

%�E���������̕���ł́H


\if0

if there exists a (classical) algorithm A that can break
a proposed cryptographic scheme (i.e., the algorithm A makes the scheme not IND-CPA et al.),
there also exists an algorithm B that solves a hard problem (factoring, DLP, shortest vector, etc.) efficiently.
It means that breaking the cryptographic scheme is harder than solving the hard problem, and the proof guarantees
the security of the scheme against attackers.
Also, estimating the hardness of solving the hard problem provides us a lower bound of the hardness of breaking the cryptographic scheme.
This is the motivation of security estimation and why we want to solve hard problems by using computers.

5. To discuss the security notions in quantum, it replaces the polynomial TM by the polynomial QTM  (of course, after the replacement, some modifications are necessary.)
With the same argument, estimating the quantum hardness of hard problems is also necessary.

6. Post-quantum cryptography is a class of cryptographic algorithms that are secure against both classical and quantum computers.
That is, the scheme must have a security reduction from the scheme to a hard problem that is thought to be not easy to solve by both classical and quantum computers.

7. The phrase "secure against quantum computers" is used by the crypto people to mean "the success probability of breaking the scheme by using
a polynomial-size quantum circuit is exponentially low."
Thus, they didn't consider any other type of quantum computers or other analog-type computers.

8. introduce representative post-quantum cryptos.

9. Applications?

\fi


\outlinecomment{somebody besides me should write this.  Moriai-san?
  Aono-san?}

\comment{Jon Dowling (RIP)'s opinions here were strong.  He believed that
  post-quantum crypto is fundamentally impossible, that all of the
  interesting asymmetric problems useful for authentication and key
  generation will ultimately fall to quantum algorithms.}

\comment{Aono: lattice crypto is very attractive because a) it reduces to
shortest vector or other problems, and b) implementation is easy,
basically a matrix times a vector~\cite{regev09:jacm}.}

\comment{learning w/ errors As + e = t mod q
candidate for post-quantum crypto}

\comment{(n.b.: cocori created a ipynb, but misunderstood the size of the
matrix necessary)}


Post-quantum cryptography is the attempt to find a public key
cryptosystem that is resistant to quantum computing, Shor's algorithm
in particular.

There is enough interest in this that the Wikipedia pages are
essentially extensive catalogs:
\url{https://en.wikipedia.org/wiki/Post-quantum_cryptography}
\url{https://en.wikipedia.org/wiki/Post-Quantum_Cryptography_Standardization}

An official-looking site:
\url{https://csrc.nist.gov/projects/post-quantum-cryptography/post-quantum-cryptography-standardization}

One recent blog posting of some use:
\url{https://blog.trailofbits.com/2018/10/22/a-guide-to-post-quantum-cryptography/}

A very recent blog posting on teaching crypto in the post-quantum
crypto age:
\url{https://news.ncsu.edu/2019/06/teaching-next-generation-cryptosystems/}
which builds on a conference presentation on the course they created:
\url{https://dl.acm.org/citation.cfm?id=3317994}
which goes into a lot of detail on crypto hardware.

A survey from a decade ago:
\url{https://www.nist.gov/publications/quantum-resistant-public-key-cryptography-survey?pub_id=901595}

Also in 2009, there was a book, which I'm sure is almost entirely
outdated by now:
\url{https://www.springer.com/jp/book/9783540887010}

Transition doc from IETF, still in draft:
\url{https://datatracker.ietf.org/doc/draft-hoffman-c2pq/}

\url{https://eprint.iacr.org/2017/847.pdf}

\url{https://arstechnica.com/gadgets/2020/07/ibm-completes-successful-field-trials-on-fully-homomorphic-encryption/}

\subsection{Notes \& References}

To be filled in eventually, mostly by moving the existing parts of
this section into this subsection!

