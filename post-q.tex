
\clearpage 

\newcommand{\memo}[1]{{\color{red} #1 }}

%%
%%	memo0813 �f�B�X�N���[�}�[���Ă��邩�ȁH
%%

\section{Post-quantum cryptography}

\memo{Add some stuffs and citations}

The post-quantum cryptography, sometimes referred as the quantum-resilient cryptography, is a wide class of cryptographic schemes which are believed to be secure against both classical and quantum computers in theory and practice.

This section introduces a motivation of post-quantum cryptography
from information theory and modern cryptographic schemes in the classical world.


\subsection{Motivation of post-quantum cryptography}

Since Shor's groundbreaking work, 
it has been known that a reasonable size quantum circuit can be used as 
an efficient oracle to solve a wide class of hidden subgroup problem,
which is an important generic problem that captures 
the integer factoring problem (ILP) and the discrete logarithm problem (DLP). 
As a direct consequence, developed quantum computers will make 
the most famous public-key cryptography such as RSA and DSA worthless.

As the countermeasure, the cryptographers have been developing 
new cryptographic schemes which they call ``post-quantum cryptography''
or ``quantum-resilient cryptography.'' 
It is widely believed that revealing secret information encrypted by a scheme is an intractable task even we have a large size of classical and quantum computers.

To discuss the resilience of cryptography to quantum computers,
we follow and extend the notion of security of cryptography in the classical world as follows.
The mathematical notion of security has been formulated by Shannon's discussion on the symmetric cryptography in \cite{Sha49}.
He defined the symmetric cryptography has the perfect security (information-theoretic security) if the eavesdropper that gets ciphertext will gain no information about the plaintext.
This security notion can be described as the equivalent condition that 
uses Turing machines (TM):
for any (unlimited time and space) TM $\mathcal{G}$ that outputs a candidate of plaintext from a given ciphertext, there exists a TM $\mathcal{G}'$ with no input that satisfies
%
\begin{equation}
	\label{eqn:ITalg}
	\Pr[ \mathcal{G}(ct) = m ] = \Pr [\mathcal{G}'() = m ].
\end{equation}
%
Here, the probability is over a fixed distribution of plaintext $m$ and key $k$, and $ct = Enc(k,ct)$.
The equivalence of probabilities means that $\mathcal{G}$ needs not to use $ct$ to recover $m$, thus, $ct$ does not contain any information about $m$ intuitively.

Shannon proved that the information-theoretic security 
requires that the sender and receiver have to share 
the secret key longer than the message itself.
The source of such unwieldiness is from the tightness of perfect security that requires the eavesdropper can get no information, or it allows unlimited power to the Turing machines.
It is easy to see any public-key cryptography does not satisfy
the perfect security \cite[Sect. 5.5.2]{GolEnc2},
and the followers tried to relax the notion.

One of the relaxed notion, which is commonly referred as the semantic security against classical computers is obtained by relaxing TM in (\ref{eqn:ITalg}) to a probabilistic polynomial TM (PPTM) and by some modifications \cite{...}, and security against quantum computes is obtained via a similar way.

\subsection{Guaranteeing the Security from the Hard Problems}

\memo{Does the content overlap with cryptanalysis section?}

As we explained above, each security notions are defined over each cryptographic system, this means that the developers have to prove their cryptographic scheme has some security property after proposing the scheme.
Historically, it has been recognized that there are two difficulties with providing the security of the scheme: one have to consider each cryptographic scheme individually, and must take into account all the feasible attacking  (albeit they are limited with a polynomial) algorithms, that makes the proof impossibly difficult.
In order to avoid the difficulty, reduction and assumption technique have been used.

The reduction is, in a nutshell, 
a mathematical technique to prove a computational problem $X$ 
is harder than another problem $Y$.
The proof has usually the following structure:
assume an algorithm $\mathcal{A}$ (working in a polynomial time) that solves $X$ (with a high probability.)
Then, provide an algorithm $\mathcal{B}$ to solve $Y$ in polynomial time with using polynomial number of calls to $\mathcal{A}$ as subroutines.
This means any entity that can solve $X$ can also solve $Y$, but the inverse does not hold.
Thus, $X$ is harder to solve than $Y$ in general.
The proof is said to be a polynomial-time reduction from $Y$ to $X$
and the notation $ X \ge_P Y$ is usually used.

In the cryptographic area, the ``security proof,'' that is, 
theorems to guarantee the security of a new cryptographic scheme,
are actually a polynomial-time reduction 
from the problem of breaking a semantic security (or other security properties) to a famous computational hard problems such as the discrete logarithm problem or the shortest vector problem.
In other words, the security proofs can guarantee that attackers must solve a hard problems to reveal secret information.
This turns out a more detailed characterization of the post-quantum cryptography, that is, a class of cryptographic schemes whose security basis is intractable for both classical and quantum computers.

Due the the reduction mechanism, the security basis of many cryptographic schemes can be unified into the hardness of a single computational problem.
However, showing the hardness of a computational problem is also a difficult problem; for example, the factoring problem is a famous widely assumed hard problem, but whether it is truly hard.
No one has been discovered a very efficient classical algorithm to solve it is a strong evidence of the hardness.

In order to concentrate the security issues, 
cryptographers introduced the notion of hardness assumption,
that is, 
assuming such computational problems are not easy to solve in practical
by using any known computers.

Some of assumptions can be breaking by theory or by practical ways.
The former case is the famous Shor's work \cite{Sh94} by using a polynomial-size quantum circuit, and the latter cases are the practical attacks for WEP or RC4 block ciphers \cite{??}.

\subsection{Classification of Post-quantum cryptographes by the base problems}

The representative post-quantum cryptographic schemes can be classified 
into several groups with the types of basis computational problems\footnote{
	They sometimes classify the scheme by the processing algorithms because two  classifications are almost overlapped.
}: 
they are all believed to be intractable by both classical and quantum computers.


\memo{Too long?}


\begin{itemize}
	\item Lattice-based cryptography is widely researched cryptographic schemes
	based on the learning with errors (LWE) problem or the shortest vector problem (SVP.)
	The first lattice-based scheme introduced by Ajtai \cite{Aj96} in 1996 was  
	the public-key cryptography, whose security basis is an approximate version of SVP.
	Since its approximation level is too large, the scheme has unfortunately been broken by Nguyen \cite{Ng97} in practice and it turns out the keys, which are represented by integer vectors $\bm{b}_1,\ldots,\bm{b}_m$ which we refer them as ``lattice basis,'' must be very large to guarantee a reasonable security level.
	
	Major of lattice-based cryptography in this decade utilizes the LWE problem introduced by Regev \cite{Reg05} in 2005 as its security basis.
	These cryptographic schemes are distinguished among the lattice-based cryptography, and are sometimes referred as the LWE-based cryptography.
	
	The basic form of LWE is the problem of finding secret vector $\bm{s} \in \mathbb{Z}_q^n$ 
	from a given matrix $A\in \mathbb{Z}_q^{m\times n}$ and a vector $\bm{b}\in \mathbb{Z}^m$, which is computed by an oracle in the following way:
	%
	\[
		\bm{b} = A\bm{s} + \bm{e} \ (\mod\ q).
	\]
	%
	Here, $\bm{e}$ is a secret vector whose coordinates are sampled from 
	a distribution $\Phi$ which outputs small integers under mod $q$.
	One wants to solve the problem can take the dimensions $n,m$ and 
	the specification of $\Phi$ as public parameters.
	Under suitable parameter settings, it has been known that the LWE problem is as hard as the SVP \cite{...}.
	
	A typical form of LWE-based cryptography uses $(A, \bm{b})$ 
	as a public key and $\bm{s}$ as the corresponding secret key. 
	For a message $m$, the ciphertext is a pair $\bm{c}_1 = \bm{r}A$ and $\bm{c}_2 = \bm{r} \cdot \bm{b} + m$ where $\bm{r}$ a randomly-selected small element vector.
	It is easy to see the receiver who knows $\bm{s}$ can recover $m$.

	From the viewpoint of computing resource, one advantage is 
	the simpleness of implementation, thus, the dominating part of encryption is the matrix-vector multiplication and it can be highly parallelized in implementation.
	On the other hand, the memory consumption to store the matrix is too large compared to the major schemes like RSA at the time, and other candidates of post-quantum cryptographies.
	
	To solve the problem, it has been proposed some compression techniques to represent the matrix by a small data.
	The ideal lattice is one of the most useful technique that represents 
	the matrix by one polynomial.
	The sender and receiver share a polynomial $f(X)$ as a public parameter, which is selected carefully, and then send a public polynomial $a(X)$ instead of $A$.
	Then, the sender decompresses it by putting $(i,j)$-element of $A$ as the coefficient of $X^j$ of $X^i a(X)\ \mod\ f(X)$.
	As the standard LWE can be reduced to SVP, this type LWE, sometimes are called the ideal-LWE problem, can also be reduced to the ideal-SVP which is an ideal analogue to the SVP.
	However, as of 2020, it is still open whether the ideal-type problems are hard enough as a basis problem to cryptosystems.

	

	\item Code-based cryptography has a longer history than the lattice-based ones.
	
		
	
	\memo{TBW}
	\item Multivariate, Isogeny, Hash

\end{itemize}



\memo{Abstract, representative basis problems (and major analysis methods?), major candidates, Solving challenges  }

%As the theorists have proven the theoretical hardness of their cryptographic scheme from reasonable assumptions, one needs to estimate practical hardness to select the practical parameters.
%\memo{SVP challenge}


\subsection{Competition in NIST/CACR}





%%-----------------------Aono 


Besides the resilience against quantum computes, 
some notable properties has been found around the cryptographies
introduced above.

The most interesting property is the homomorphic encrypton.

\memo{Homomorphic encryption, but this is independent from quantum-resilient}


\outlinecomment{somebody besides me should write this.  Moriai-san?
  Aono-san?}

\comment{Jon Dowling (RIP)'s opinions here were strong.  He believed that
  post-quantum crypto is fundamentally impossible, that all of the
  interesting asymmetric problems useful for authentication and key
  generation will ultimately fall to quantum algorithms.}

\comment{Aono: lattice crypto is very attractive because a) it reduces to
shortest vector or other problems, and b) implementation is easy,
basically a matrix times a vector~\cite{regev09:jacm}.}

\comment{learning w/ errors As + e = t mod q
candidate for post-quantum crypto}

\comment{(n.b.: cocori created a ipynb, but misunderstood the size of the
matrix necessary)}


Post-quantum cryptography is the attempt to find a public key
cryptosystem that is resistant to quantum computing, Shor's algorithm
in particular.

There is enough interest in this that the Wikipedia pages are
essentially extensive catalogs:
\url{https://en.wikipedia.org/wiki/Post-quantum_cryptography}
\url{https://en.wikipedia.org/wiki/Post-Quantum_Cryptography_Standardization}

An official-looking site:
\url{https://csrc.nist.gov/projects/post-quantum-cryptography/post-quantum-cryptography-standardization}

One recent blog posting of some use:
\url{https://blog.trailofbits.com/2018/10/22/a-guide-to-post-quantum-cryptography/}

A very recent blog posting on teaching crypto in the post-quantum
crypto age:
\url{https://news.ncsu.edu/2019/06/teaching-next-generation-cryptosystems/}
which builds on a conference presentation on the course they created:
\url{https://dl.acm.org/citation.cfm?id=3317994}
which goes into a lot of detail on crypto hardware.

A survey from a decade ago:
\url{https://www.nist.gov/publications/quantum-resistant-public-key-cryptography-survey?pub_id=901595}

Also in 2009, there was a book, which I'm sure is almost entirely
outdated by now:
\url{https://www.springer.com/jp/book/9783540887010}

Transition doc from IETF, still in draft:
\url{https://datatracker.ietf.org/doc/draft-hoffman-c2pq/}

\url{https://eprint.iacr.org/2017/847.pdf}

\url{https://arstechnica.com/gadgets/2020/07/ibm-completes-successful-field-trials-on-fully-homomorphic-encryption/}

\subsection{Notes \& References}

To be filled in eventually, mostly by moving the existing parts of
this section into this subsection!

