\section{Post-quantum cryptography}

\outlinecomment{somebody besides me should write this.  Moriai-san?
  Aono-san?}

\comment{Jon Dowling (RIP)'s opinions here were strong.  He believed that
  post-quantum crypto is fundamentally impossible, that all of the
  interesting asymmetric problems useful for authentication and key
  generation will ultimately fall to quantum algorithms.}

\comment{Aono: lattice crypto is very attractive because a) it reduces to
shortest vector or other problems, and b) implementation is easy,
basically a matrix times a vector~\cite{regev09:jacm}.}

\comment{learning w/ errors As + e = t mod q
candidate for post-quantum crypto}

\comment{(n.b.: cocori created a ipynb, but misunderstood the size of the
matrix necessary)}


Post-quantum cryptography is the attempt to find a public key
cryptosystem that is resistant to quantum computing, Shor's algorithm
in particular.

There is enough interest in this that the Wikipedia pages are
essentially extensive catalogs:
\url{https://en.wikipedia.org/wiki/Post-quantum_cryptography}
\url{https://en.wikipedia.org/wiki/Post-Quantum_Cryptography_Standardization}

An official-looking site:
\url{https://csrc.nist.gov/projects/post-quantum-cryptography/post-quantum-cryptography-standardization}

One recent blog posting of some use:
\url{https://blog.trailofbits.com/2018/10/22/a-guide-to-post-quantum-cryptography/}

A very recent blog posting on teaching crypto in the post-quantum
crypto age:
\url{https://news.ncsu.edu/2019/06/teaching-next-generation-cryptosystems/}
which builds on a conference presentation on the course they created:
\url{https://dl.acm.org/citation.cfm?id=3317994}
which goes into a lot of detail on crypto hardware.

A survey from a decade ago:
\url{https://www.nist.gov/publications/quantum-resistant-public-key-cryptography-survey?pub_id=901595}

Also in 2009, there was a book, which I'm sure is almost entirely
outdated by now:
\url{https://www.springer.com/jp/book/9783540887010}

Transition doc from IETF, still in draft:
\url{https://datatracker.ietf.org/doc/draft-hoffman-c2pq/}

\subsection{Notes \& References}

To be filled in eventually, mostly by moving the existing parts of
this section into this subsection!

